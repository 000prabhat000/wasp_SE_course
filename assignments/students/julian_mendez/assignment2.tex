\documentclass[11pt]{article}


\begin{document}

    \title{Software Engineering and Cloud Computing \\
    Essay
    }
    \author{Julian Alfredo Mendez \\
    Ume{\aa} University}

    \date{September 2022}

    \maketitle


    \section{Current Research}

    This essay starts by explaining my current research, then explains my view on three relevant topics of Software Engineering, and ends with a discussion about future trends and directions of Software Engineering.

    My current research is centered on techniques and tools to formalize ethical requirements for Artificial Intelligence (AI) systems.
    My main focus is bridging the gap between what a human intends to communicate and what a computer interprets from that.
    My main interest is on ethical requirements, which could be sometimes either too vague to express in a formal language, or too technical to be grasped by most humans.

    The first steps in my research have been to develop a language that preserves most of the expressive power of traditional computer languages, but that keeps a very clean way to express concepts.
    On top of this language, I am developing a toolbox to connect components to express ethical concepts.

    The language addresses clarity by limiting how descriptions can be built to favor readability, and it can be integrated with a proof checker.
    The language does not pursue defining what is ethical and what is not, neither does it try to formalize all ethical requirements.
    Its main purpose is to be useful to formalize some relevant ethical issues, and to bring extra reliability in parts that can be formally proved.


    \section{Relevant Topics}

    In the following, I discuss three topics which I consider relevant to my research.

    \subsection{Automated Software Testing}

    Automated software testing refers to the use of automation to execute tests on a piece of software.
    Testing has always been relevant for software development, but its value became especially visible after the appearance of Test Driven Development in 2003, based on the principles of Extreme Programming.
    This happened not much later than when the Year 2000 Problem (Y2K) affected the maintenance costs of expensive computer systems, like those for banking, aeronautics, and machinery control.

    Nowadays, the need of having maintainable and flexible software makes testing essential.
    Good tests and high test coverage allow refactoring, which in turn leads to better designs.
    This topic is indirectly linked to my research, because testing usually contributes to improve desired ethical requirements in a system, like transparency and robustness.

    \subsection{Architecture and Design}

    Software architecture and design refer to a general view of fundamental parts of software systems.
    Architecture and design are critical in software development, since a wrong architecture or design can make a system extremely expensive or simply unfeasible.

    Modern development tools are very helpful for refactoring, which combined with automated tests, can help to adapt and to improve a design, making a system more durable and reliable.
    This topic is related to my research, because the language I develop has specific constructs to simplify modeling of a class-based design, with the purpose of having designs that many stakeholders can understand.

    \subsection{Security and Privacy}

    Data security and data privacy are two different concepts, but they share common ideas.
    The former focuses on protecting data from unauthorized access, either for modification or just retrieval.
    The latter focuses on protecting private information of individuals, in such a way that they can choose what, when, and with whom sharing that information.

    I consider that both concepts are related to my research.
    In particular, privacy is object of high interest at a societal level, especially considering that some large providers of online services can monetize private information of their users, and in some cases, use it to influence their opinion.
    Formalizing some cases to model how a system can respect privacy is one of the planned use cases of my research.


    \section{Future Trends and Directions of Software Engineering}

    The future is clearly uncertain, but there are already some trends that could be highlighted.
    In this section, I express my perception of the relation between my research topic and the evolution of Software Engineering.

    My impression is that future Software Engineering will help to produce more transparent and robust software.
    Aside from transparency and robustness, future software systems will include other ethical requirements to ensure societal values, especially in aspects concerning fairness and privacy.

    We can observe that over the last years hardware producers have brought computers with more microprocessors instead of faster ones.
    This fosters programming that includes techniques for concurrent execution, for example by using immutable structures.

    Being a distinctive trait of functional programming, immutability helps to preserve a structure across executions without changing it.
    The language I present in my research is a simplified statically typed purely functional language, which induces to create small functions that use immutable structures.

    Immutability helps to prevent unexpected side effects and static typing to prevent common type mistakes.
    Both features lead to a more maintainable system, with cleaner source code and a better design.

    The language I develop has the potential of proving the correctness of some pieces of code by using a proof checker.
    Even if doing this can require significant effort, this effort can be justified for critical pieces of code.
    I believe that future software will use more libraries with pieces of proven code, as current software uses libraries with very high test coverage.

    Regarding the relation between Machine Learning (ML) and Software Engineering, I consider that ML has to comply with ethical requirements.
    An image recognition system that distinguishes between dogs and wolves by using the background and an autonomous vehicle that sometimes interprets a Stop sign like a Speed Limit 45 mph (72 km/h) sign are examples of unreliable systems that cannot be in production.

    ML systems have currently some limitations and they should be properly contained, if society needs to trust them.
    Overall, I consider that my research is aligned with the aforementioned trends, and helps bridging the ethical gap between humans and machines.

\end{document}
