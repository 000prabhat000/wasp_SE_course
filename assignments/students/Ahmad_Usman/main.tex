
\documentclass{article}
\usepackage[a4paper, total={6in, 8in}]{geometry}
\title{SE: Reflection on Pynblint}
\author{Ahmad B. Usman  (ahmad.usman@liu.se)}



\begin{document}
\maketitle
\section{About Pynblint}
Lugi Quaranta and I together went through the installation and a practical walkthrough of the Pynblint tool. During this session, I have looked at my notebooks coding from a different perspective. I learned that Pynbint is like a debugger tool for a notebook, that allows programmers to figure out issues related to their code, starting by pointing out some suggestions, hints and some errors. We then briefly discussed the static analysis for Machine learning. 
\section{Pynblint Usefulness}
I’m currently  experimenting with Pynblint in one of the courses I’m participating in, which excessively uses notebooks. I understand however how important it is to synthesize our code and get some feedback on how we can make an improvement on the code. In the future however, I’m not sure if I will personally be using Pynblint, probably if I come across other courses that use a notebook. As of now, the tool did not attract users due to the lack of popularity, however, I consider it to be useful in the future and bring tremendous advantages to both academia and industry.

\section{Pynblint Improvement}

There might be several ways to improve Pynblint, either by adding new features that can enable users to automatically resolve the issues and print out the lines that have been corrected, or  provide possible solutions. A completely  different direction was to involve several programming languages other than python or  notebook (languages such as C C++ and JAVA) or enable cross-compiling features or at least make it compatible  with some Integrated Development Environment (IDE).


\section{Limitation on Static Analysis Tools}

Among possible limitations is that the static analysis tool  does not analyze the actual and entire ML code. This limitation will engender hesitation among users to completely rely on this static analysis tool. It could be more efficient when analyzing a small portion of code, however when dealing with a large scale ML data it will even become difficult to follow up with the outcome.  Another limitation, as I mentioned in regard to Pynblint, the static analysis tool doesn’t also provide corrections possibilities, it only provides transparent feedback.

\end{document}